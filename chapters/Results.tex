\chapter{Results}\label{chapter:Results}
Present the results from following the procedures in \nameref{chapter:Method}, note some observations, but do not discuss them.

\begin{table}[h]
    \centering
    \begin{tabular}{l | l | l}
    A & B & C \\
    \hline
    1 & 2 & 3 \\
    4 & 5 & 6
    \end{tabular}
    \caption{very basic table}
    \label{tab:abc}
    \end{table}

\section{Simulations with 'cheap FSI'}

\begin{figure}
    \centering
    \begin{subfigure}[b]{0.45\textwidth}
        \centering
        \includegraphics[width=\textwidth]{results/4-HV_1p5_T6_15}
        \caption{\textcolor{tab:red}{\underline{4-HV}}\_1p5\_T6\_15}
        \label{fig:4-HV_1p5_T6_15}
    \end{subfigure}
    \begin{subfigure}[b]{0.45\textwidth}
        \centering
        \includegraphics[width=\textwidth]{results/4-45_1p5_T6_15}
        \caption{\textcolor{tab:red}{\underline{4-45}}\_1p5\_T6\_15}
        \label{fig:4-45_1p5_T6_15}
    \end{subfigure}
    \begin{subfigure}[b]{0.45\textwidth}
        \centering
        \includegraphics[width=\textwidth]{results/1-HV_1p5_T6_15}
        \caption{\textcolor{tab:red}{\underline{1-HV}}\_1p5\_T6\_15}
        \label{fig:1-HV_1p5_T6_15}
    \end{subfigure}
    \begin{subfigure}[b]{0.45\textwidth}
        \centering
        \includegraphics[width=\textwidth]{results/1-22p5_1p5_T6_15}
        \caption{\textcolor{tab:red}{\underline{1-22p5}}\_1p5\_T6\_15}
        \label{fig:1-22p5_1p5_T6_15}
    \end{subfigure}
    \caption{Effect of the slits.}
    \label{fig:effect_slit}
\end{figure}

\begin{figure}
    \centering
    \begin{subfigure}[b]{0.45\textwidth}
        \centering
        \includegraphics[width=\textwidth]{results/1-HV_3p0_T6_15}
        \caption{1-HV\_\textcolor{tab:red}{\underline{3p0}}\_T6\_15}
        \label{fig:1-HV_3p0_T6_15}
    \end{subfigure}
    \begin{subfigure}[b]{0.45\textwidth}
        \centering
        \includegraphics[width=\textwidth]{results/1-HV_1p5_T6_15_mesh}
        \caption{\text{1-HV\_\textcolor{tab:red}{\underline{1p5}}\_T6\_15}}
        \label{fig:1-HV_1p5_T6_15_mesh}
    \end{subfigure}
    \caption{Effect of different element sizes.}
    \label{fig:effect_element}
\end{figure}

\begin{figure}
    \centering
    \begin{subfigure}[b]{0.32\textwidth}
        \centering
        \includegraphics[width=\textwidth]{results/1-22p5_1p5_T4_15}
        \caption{\text{1-22p5\_1p5\_\textcolor{tab:red}{\underline{T4}}\_15}}
        \label{fig:T4}
    \end{subfigure}
    \begin{subfigure}[b]{0.32\textwidth}
        \centering
        \includegraphics[width=\textwidth]{results/1-22p5_1p5_T6_15}
        \caption{\text{1-22p5\_1p5\_\textcolor{tab:red}{\underline{T6}}\_15}}
        \label{fig:T6}
    \end{subfigure}
    \begin{subfigure}[b]{0.32\textwidth}
        \centering
        \includegraphics[width=\textwidth]{results/1-22p5_1p5_T7_15}
        \caption{\text{1-22p5\_1p5\_\textcolor{tab:red}{\underline{T7}}\_15}}
        \label{fig:T7}
    \end{subfigure}
    \caption{Effect of the materials.}
    \label{fig:effect_material}
\end{figure}

\begin{figure}
    \centering
    \begin{subfigure}[b]{0.45\textwidth}
        \centering
        \includegraphics[width=\textwidth]{results/1-HV_1p5_T6_10}
        \caption{1-HV\_1p5\_T6\_\textcolor{tab:red}{\underline{10}}}
        \label{fig:1-HV_1p5_T6_10}
    \end{subfigure}
    \begin{subfigure}[b]{0.45\textwidth}
        \centering
        \includegraphics[width=\textwidth]{results/1-HV_1p5_T6_15}
        \caption{1-HV\_1p5\_T6\_\textcolor{tab:red}{\underline{15}}}
        \label{fig:1-HV_1p5_T6_15_amp}
    \end{subfigure}
    \caption{Effect of different pressure amplitudes.}
    \label{fig:effect_amp}
\end{figure}

\section{ANN study with 'cheap FSI'}

\begin{figure}
    \centering
    \includegraphics[width=\textwidth]{results/para1_net}
    \caption{Final training losses for different network architectures defined by activation function, number of hidden layers and number of neurons per hidden layer.}
    \label{fig:para1_net}
\end{figure}

\begin{figure}
    \centering
    \includegraphics[width=\textwidth]{results/para1_all}
    \caption{Bottom text}
    \label{fig:para1_all}
\end{figure}

\begin{figure}
    \centering
    \includegraphics[width=\textwidth]{results/para1_test}
    \caption{Bottom text}
    \label{fig:para1_test}
\end{figure}

\begin{figure}
    \centering
    \includegraphics[width=\textwidth]{results/para2_all}
    \caption{Bottom text}
    \label{fig:para2_all}
\end{figure}

\begin{figure}
    \centering
    \includegraphics[width=\textwidth]{results/para3_all}
    \caption{Bottom text}
    \label{fig:para3_all}
\end{figure}

\begin{figure}
    \centering
    \includegraphics[width=\textwidth]{results/para3_test}
    \caption{Bottom text}
    \label{fig:para3_test}
\end{figure}

\begin{figure}
    \centering
    \includegraphics[width=\textwidth]{results/para4_all}
    \caption{Bottom text}
    \label{fig:para4_all}
\end{figure}

\begin{figure}
    \centering
    \includegraphics[width=\textwidth]{results/para4_test}
    \caption{Bottom text}
    \label{fig:para4_test}
\end{figure}

\begin{figure}
    \centering
    \includegraphics[width=\textwidth]{results/para_all}
    \caption{Bottom text}
    \label{fig:para_all}
\end{figure}

\begin{figure}
    \centering
    \includegraphics[width=\textwidth]{results/vars}
    \caption{Bottom text}
    \label{fig:para_vars}
\end{figure}

\begin{figure}
    \centering
    \includegraphics[width=\textwidth]{results/para5_all}
    \caption{Bottom text}
    \label{fig:para5_all}
\end{figure}

\begin{figure}
    \centering
    \includegraphics[width=\textwidth]{results/para3_vs_para5}
    \caption{Bottom text}
    \label{fig:para3_vs_para5}
\end{figure}